\documentclass[10pt, twoside]{article}
\usepackage[margin=1in]{geometry}
\usepackage{setspace}
\usepackage{titlesec}
\usepackage{fancyhdr}
\usepackage{textcase}
\usepackage{parskip} 
\usepackage{enumitem}
\usepackage[colorlinks=true]{hyperref}

\geometry{
    paperwidth=6in,  % Adjust the width of the paper
    paperheight=9in,   % Adjust the height of the paper
    margin=0.85in,
}

% Line spacing: tighter than double
\setstretch{1.2}

\setlength{\parindent}{1.5em}
\setlength{\parskip}{0pt}

\title{Concerning the Nature of Diagnosis}
\author{Eric Berne, M.D.} 

\makeatletter
\let\thetitle\@title
\let\theauthor\@author
\makeatother

% Header style
\pagestyle{fancy}
\fancyhf{}
\fancyhead[LO,RE]{\thepage}
\fancyhead[CO]{\scriptsize\MakeTextUppercase{Concerning the Nature of Diagnosis}}
\fancyhead[CE]{\scriptsize\MakeTextUppercase\theauthor}

\renewcommand{\headrulewidth}{0pt}
\thispagestyle{empty}

% Section headers
% \titleformat{\section}[block]{\normalsize\bfseries}{}{0em}{}
\titleformat{\section}
  {\normalfont\centering\small}
  {\Roman{section}.}
  {0.5em}
  {\MakeUppercase}

% Custom maketitle with slightly bold, uppercase title
\renewcommand{\maketitle}{
  \begin{center}
    {\fontseries{bx}\normalsize\MakeTextUppercase\thetitle \par}
    {\footnotesize\MakeTextUppercase{by \theauthor}}
  \end{center}
}

\newcommand{\sref}[1]{\textsuperscript{\ref{#1}}}

\begin{document}

\maketitle

\section{Diagnosis by Inspection}

Every human being is to some extent capable of establishing a diagnosis by inspection. Even mentally dull individuals can distinguish between a 10-year-old child and a 20-year-old man. Most adults can work with a smaller difference threshold, such as distinguishing between a 30-year-old and a 40-year-old in the majority of trials. A clinician likes to think he can do better than that. Some physicians become very skillful in this respect, and can usually guess the age of a patient under 40 accurately to within two years. The writer has seen talented experts repeatedly conjecture adults' ages correctly to within three months. Age-guessers at carnivals are willing to bet on their accuracy, and come out very well in large series of cases. 

There are several aspects to such diagnostic processes which are worth discussing, particularly the preverbal functions involved. Perhaps the simplest example is that of a dull individual who reveals by his behavior that he distinguishes between a 10-year-old and a 20-year-old. In such a case there are several possibilities.

\begin{enumerate}[leftmargin=0pt,itemindent=3em,labelsep=0.5em]
    \item The individual may not be aware that he is judging the subjects' ages and guiding his behavior accordingly. If he is asked, he may even reply in the negative. That is, he is not even aware that he knows something. 
    \item If he becomes aware that he is behaving differently towards the two subjects, he may not know why. That is, he becomes aware that he knows something but does not know what it is that he knows. 
    \item If he realizes that he is treating the two subjects differently because of the difference in their ages, then he knows that he knows something about them, and knows what it is that he knows, but he may still be unable to say how he knows it. 
    \item If he tries to explain how he knows it, his explanation is likely to be unsatisfactory both to him and to the listener. 
\end{enumerate}

In this case, the individual behaves as though he had made a diagnosis, but he may be quite unaware that he has made it. If he is aware, he may be unaware of what it is that he has diagnosed; if he is aware of that, he may be unable to justify his diagnosis, and if he attempts to do so, his justification may not be convincing. 

A more subtle example in some ways is that of the clinician who decides before asking that a certain patient is 36 years old. This clinician knows that he knows something, he knows what it is that he knows, and he is able to say in a general way how he knows it. He may say that his diagnosis is based on careful inspection of the individual's face; or he may say that it is the fruit of free-floating, undirected attention. If he is asked, however, to formulate his criteria for differentiating a 30-year-old from a 39-year-old, he will usually confess his inability to do so. 

A carnival weight-guesser recently tried to formulate his diagnostic criteria (Mygatt 1947\sref{ref:ref1_mygatt1947}). These criteria may be useful within a 50-pound margin of permissible error, but they give no convincing indication of how he could guess weights accurately to within five pounds, as these people can readily do with a very high percentage of accuracy. His exposition seems to be a rationalization of his diagnoses rather than a basis for them. 

It is apparent, therefore, that there are cognitive processes which function below the level of consciousness. In fact, human beings, when they are in full possession of their faculties, behave at all times as though they were continually and quickly making very subtle judgments about their fellowmen without being aware that they are doing so; or if they are aware of what they are doing, without being aware of how they do it. If they try to verbalize these processes, their explanations are too crude to account for the refinement of the judgments, even among professionals who are continually trying to formulate diagnostic criteria. Their explanations seem to be rationalizations of their diagnoses, rather than expositions of the real diagnostic process. 

Psychiatrists, for example, are continually making accurate diagnoses all over the world, yet it is a noteworthy accomplishment for one of them to formulate in words a useful diagnostic criterion; even then, such a criterion has to be hedged with cautions and exceptions. Although each such formulation may be helpful in a small way, in essence the preliminary diagnosis of an experienced clinician is a product of preverbal processes which are functions of his skill, keenness, and experience, rather than the result of the deliberate application of a collection of formal criteria. It appears that the most important judgments which human beings make concerning each other are the products of preverbal processes - cognition without insight - which function almost automatically below the level of consciousness. In the case of age diagnosis and in many other cases as we'll, it can be readily shown that the subconscious processes are more accurate, refined, and reliable than conscious diagnostic processes. Nevertheless, a certain restraint deprives many people of the free use of this valuable faculty.

\section{Verbalizing Diagnostic Criteria}

During the last war Professor Eugen Kahn, in a private conversation, made a remark to the effect that it is possible for an alert and experienced psychiatrist to pick out men unfit for military service very rapidly in the setting of an induction center, and that detailed study of inductees would add comparatively little to such a psychiatrist's percentage of accuracy. This statement was given a modest test by the writer at the first opportunity, and it was found that after a preliminary period of practice it was possible to make certain judgments in a minimum period of time with considerable accuracy. 

A preliminary study was made at an induction center where the clerical staff administered the Cornell Selectee Index. It was found possible to estimate closely in a large number of cases the score of each candidate by looking at him for a few seconds before turning over the sheet to see what his score was. It was found that the estimates were correct to within three points in a large percentage of cases, even including those whose scores lay between 15 and 25. This was taken to mean that the inductee revealed by a few seconds of spontaneous behavior in the specific situation characteristics which were correlated with his score on the questionnaire. 

After this \dots [the previously described] attempt was made to predict, after a few moments of looking in each case, the responses of [the] several thousand soldiers to the \dots two questions:
\begin{enumerate}
    \item Are you nervous? 
    \item Have you ever been to a psychiatrist?
\end{enumerate}

\dots In a rough way this was a guess as to ``how neurotic" the individual was. It was found after a few weeks when a plateau of accuracy had been attained that the guesses were uniformly correct, with only a few exceptions: that is, in a few cases where the guess was ``yes," the answer was ``no"; and in a few cases where the guess was ``no," the answer was ``yes." (When a similar and more carefully controlled [written] series was discussed with the late Dr. Paul Federn, he made the surprising remark that the percentages of error in both directions seemed high to him.) Since the records were incomplete, no figures are offered, nor are they essential to the spirit of this study. 

An attempt was made to formulate the criteria upon which the predictions were based. It seemed evident that the formulation was not completely successful, for the percentage of such correct predictions remained higher when the ``intuitive" process was allowed to function without conscious interference than when judgments were attempted on the basis of deliberate use of the criteria which had been verbalized. The conclusion was drawn that the criteria used in the ``intuitive" process had not all been formulated. Those which were formulated, however, could be applied deliberately with much success. The attempt to formulate the nature of the intuitive process itself has [already] been described\dots . 

The verbalized diagnostic criteria formed the basis for dividing the soldiers into three groups. 

\begin{enumerate}
    \item One group of soldiers would meet the psychiatrist's eye squarely and answer the ensuing questions in a clear, firm voice. Many of them would smile at him before the questions were asked and a good many of them after answering, in an engaging and friendly fashion, which provoked the designation of ``a well-integrated reaction." The psychiatrist, having just spent a long tour of duty dealing with hospitalized neurotics, had the subjective impression: ``These men have nothing to hide," which was in contrast to his impression of the neurotics, who retrospectively began to seem ``as though they had something to hide." 
    
    During the few seconds of silent observation, the men in this group sat in a fairly relaxed fashion without making purposeless movements. They nearly all answered the two questions negatively, and the examiner began to consider that men who behaved in this way were ``normal." Since they had all been through one to five years of military service and stated that they did not feel ``nervous" at the end and had not consulted or been sent to a psychiatrist during this period, they were loosely considered - at face value - to have demonstrated adequate personality integration in the military situation. 

    \item A second category of men, when they sat down and were greeted with silence, would make ``purposeless" movements, usually of a rhythmic nature, such as swinging their crossed legs; or their gaze would wander to and from the examiner and finally become fixed on some nearby object or some part of their own bodies; or they would finally meet the examiner's eyes in an uncertain fashion. This type of reaction nearly always involved movements of decreasing amplitude, slowly coming to rest, so that they gave the impression of being ``pendular" in quality. 
    
    Men in this category also usually gave negative replies to the two stock questions. It was noted after some study that positive replies came from this group nearly always when ``the pendulum took longer than usual to come to rest," so that this type of reaction had to be considered in four dimensions rather than in three.
    
    Other positive replies came when the movements were observed to be of unusual amplitude. 

    \item A third group of men reacted to the situation by making ``purposeless," apparently largely ``unconscious," movements, which differed from those observed in the previous group in lacking the ``pendular" quality. Some of them tried to meet the psychiatrists gaze, lowered their eyes, and began to tap with their fingers on their thighs or on the desk, or to twist their hands and fingers. Some of them looked around the room and never attempted to meet the psychiatrist's eye. Some of them would start to smile in an uncertain fashion, change their minds, lower their gaze, and begin to fidget or squirm. Most of them did not even attempt to smile; those who did faltered. In other cases, tremors would begin or become worse if present initially. They usually did not answer questions in a clear, firm voice but in a restrained, weak, timid tone.
    
    Members of this group usually answered either one or both of the stock questions in the affirmative, and there appeared to be a correlation between the amount of energy expended ``purposely" and the number of questions answered in the affirmative. The more energy they expended in this way, the more likely they were to say that they had been to a psychiatrist, even when they said that they were not ``nervous" at the moment.
    
    In some cases where no such ``purposeless" expenditures of energy were obvious, the psychiatrist mentally placed the men in the first category and found that he had made an error; i.e., some men who did not appear to make ``purposeless" expenditures of energy did say that they were nervous or had been to a psychiatrist. Many of these said they had consulted the psychiatrist for ``stomach trouble." Closer observation revealed that these men were expending energy ``purposelessly," not by overt movements but through an abnormal increase in muscular tonus; i.e., they were sitting still but they were very tense.
\end{enumerate}
 

The utilization of energy by these men, especially those in the first and third groups, who under the clinical circumstances constituted the ``normal" and the ``neurotic" groups respectively, is worth considering briefly. 

In the ``normal" group the energy activated by the stimulus of the interview was all or nearly all channelized into useful and well-integrated activity, giving force and firmness to the voice. Part of it in many cases found expression in a friendly smile. Little or none of it was manifested in ``purposeless" activity. These men were able to communicate directly with the examiner. 

In the ``neurotic" group, a large part of the activated energy was ``misdirected" during the preliminary period of sitting. This ``neurotic" activity appeared to be purposeless and autistic, neither friendly nor hostile. But to a professional observer it inadvertently communicated anxiety or hostility. Psychologically, the ``purposeless" movements seemed to be distorted kicks, blows, and hostile oral grimaces. 

It is noteworthy that in both groups the facial muscles might come into play. Among the ``normals," the facial muscles were used as overt instruments of a frank ``object relationship," that is, to communicate a friendly feeling to a new human being by smiling. Among the ``neurotics," these muscles were used in an autistic way; e.g., for mouth twitchings, which did not directly communicate anything to the other person concerned, except insofar as he was a professional observer.

On another plane of verbalization, the ``normals" gave the impression of being happy and of having nothing to hide, while the ``neurotics" gave the impression of being unhappy and of having something to hide. 

Quantitatively speaking, the psychiatrist would say that the meeting with him was related to a certain expenditure of energy. In ``normal" individuals, this found free and more or less full expression in organized, efficient, and tension-relieving activity. In ``neurotic" individuals, for reasons which were specific to each individual, it did not, and instead manifested itself in disorganized, inefficient, and unsatisfying activity. In the intermediate group, anxiety also took precedence, and among the more neurotic members of this group it presented a problem which could not be solved in the allotted time. 

Further discussion along these lines would go beyond the present scope. It must be emphasized that these soldiers were reacting to a certain individual who greeted them in a certain way. Another individual who greeted them in another way would certainly obtain a different series of responses, as would the same individual who greeted them in another way. In the opinion of the writer the essence of the situation was the neutral, unswerving gaze which left the soldier for the first few seconds without any indication of what he was supposed to do. 

\section{Diagnosis as a Configurational Process}

This experience in the rapid ``diagnosis" of large numbers of individuals demonstrated that the intuitive processes which function below the level of consciousness are susceptible of study under certain conditions. Such subconscious processes of cognition are continually at work in human beings and are of importance in everyday living and of theoretical interest in professional work. Biologically, intuition may be related to primitive cognitive processes in lower animals (Darwin 1886\sref{ref:ref2_darwin1886}; Krogh 1948\sref{ref:ref3_krogh1948}; Wiener 1948b\sref{ref:ref4_wiener1948_time}). Phylogenetically, it preceded verbal knowledge and communication (Sturtevant 1947, p. 48\sref{ref:ref5_sturtevant1947}), and ontogenetically as well (Deutsch 1944\sref{ref:ref6_deutsch1944}; Schilder 1942, p. 247\sref{ref:ref7_schilder1942}). Psychologically, it is important because it is related to problems of group behavior and their limiting case, ``what happens between two people," which is the nuclear problem of everyday living and of psychotherapeutic technique. It must be taken into account in formulating the psychological aspects of communication theory, which are becoming ever more important as mathematicians and engineers need assistance from psychology in elaborating their theories of communication (Brillouin 1950\sref{ref:ref8_brillouin1950}). 

Of the many ramifications of this subject, only three will be considered here. First, why is it that such important cognitive processes have been so little studied? The modern literature is limited, consisting mainly of the work of H. Bergson (1944\sref{ref:ref9_bergson1944}), C. G. Jung (1946\sref{ref:ref10_jung1946}), T. Reik (1948\sref{ref:ref11_reik1948}), and K. W. Wild (1938\sref{ref:ref12_wild1938}). It is noticeable that many people find intuition a disagreeable topic to discuss, and their attitudes may have some of the qualities of a dynamic resistance. Bergson has discussed such attitudes from the philosophical point of view, while W. Kohler (1929\sref{ref:ref13_kohler1929}) has discussed in another connection similar attitudes rationalized on methodological grounds. People often act as though they are afraid to admit that they know something when they cannot explain to themselves exactly how they know it. We humans sometimes feel more secure if we deny the existence of certain cognitive processes, when we cannot convince ourselves that we have complete insight into them. 

Secondly, it is possible to study the development of intuitive processes in a particular field, and a systematic research into this aspect of the matter might be undertaken. In the case of psychiatric diagnosis, certain statements can be made. There are three phases in the development of this skill. With beginners, it is an additive process based on didactic criteria: a series of consciously directed observations is consciously sorted in accordance with formal notions, in an attempt to find a close match among preconceived schemes, in such a way that the whole process can be verbalized for the benefit of the teacher. With experienced clinicians, both the process of observing and the process of sorting and matching are more plastic and complex, and take place partly below the threshold of consciousness. This is the psychologic difference between an amateur and a professional. In the third phase, which verifies and controls the others, the diagnosis is first made plastically below the level of consciousness, and later a secondary process of verbalization and rationalization takes place. 

Thirdly, verbal processes are additive, while intuitive processes are integrative. The student in a new field knows no more and no less about that field than he is able to verbalize in the first place. He states his observations and synthesizes them into a diagnosis. The experienced clinician, on the other hand, verbalizes certain aspects which he analytically dissects out of his initial intuitive picture of the patient, which is largely a function of his past experience in the field. In psychologic terms, the student builds up a mosaic, while the experienced clinician tears down a configuration. 

The configurational quality of the preliminary intuitive diagnosis can be demonstrated in various ways. In whatever manner the diagnostic processes are categorized, one factor remains constant: these processes are based on observation of the patient or, more precisely, on the understanding of his communications. Many clinicians will agree that the more direct the communications, the more accurate the preliminary diagnosis will be. Most clinicians feel more confident after seeing a new patient in the office than after talking to him ``indirectly" over the telephone. Similarly, they prefer to make a diagnosis by interview rather than by mail, even in the case of psychotics. Evidently something happens during various processes of filtration which make many clinicians feel more or less insecure even with a considerable amount of such indirect communication at their disposal. In effect, the clinician tends to feel most secure in his diagnosis when he can observe the total configuration of the patient's personality without the intervention of filters. 

In such examples, the filtration processes are obvious to those concerned. The clinician is fully aware that something has been subtracted from the total configuration of the patient's behavior by the communication through an extraneous medium. Not all clinicians, however, are fully aware of the internal systems which may filter their observations and subtract from the material available for the diagnostic processes. The student, for example, may not be actively aware how much his observations are being filtered through the schemata he has learned from teachers and books'. Many more experienced clinicians may not be fully cognizant of every aspect of the scheme of countertransference, anxiety, and self-mistrust which nature places between us and our patients. Under such conditions, the diagnostic processes have to work partly with indirect communications strictly speaking. This may account for some of the consistent trends among the participants in psychiatric staff conferences, where one individual may habitually lean toward the diagnosis of depression, another may favor the diagnosis of schizophrenia, while a third may emphasize ``psychopathy." Each experienced participant is usually right somehow or other, and obviously each filters differently what he observes. The point here is that he also filters what he is trying to tell himself; that is, intuition has to filter through an arranging ego. 

A more specific example of the configurational nature of the intuitive diagnostic process is contained in the following observation. The writer has observed that, when he works in a new office, he is not so sure of himself in the preliminary diagnostic phase as he is in his old office (the writer has also observed that there are three major phases in changing the ego boundaries to include a new environment, and that these phases develop after 5 or 6 days, 39 to 45 days, and about 180 days, respectively, in his own case and in many other cases which he has studied in this regard. But this is a different aspect of the problem). It is as though he had to see the patient against a background which had become standardized in his mind before he felt confident that he understood the case. Thus the patient is not the whole configuration concerned in the diagnostic process but is part of a configuration, and it is easier to understand him when he is seen against a well-known background and makes his communications from this background. It appears that this background is an important factor in the more subtle aspects of diagnosis. Yet in defending the diagnosis verbally, the role of the background can only incidentally be formulated. It may be mentioned how the patient used the background, what remarks he made about it, and so forth, but actually the significant diagnostic factor is not the formulation of how the patient uses the background but the subconscious perception of how he fits into it as part of a configuration. The diagnosis seems to be made ``for me" by some cultivated faculty which operates on the whole configuration below the level of consciousness; what is formally called ``making the diagnosis," that is, explaining the grounds for it, is only a secondary additive process, justifying what is partly known in some other way, through preconscious and unconscious cognitive process. 

Careful consideration reveals another interesting fact. The subconscious process does not really make a diagnosis. It makes a preverbal judgment of the configuration, knowing nothing of diagnostic terminology. What happens is that this judgment is verbalized in diagnostic terminology. For example, [when] the writer \dots tried to guess by inspection the occupations of \dots [the soldiers, he] was quite successful in picking out farmers and mechanics. Careful study revealed, however, that the subconscious process was actually not judging ``occupation," but the attitudes of the men toward reality problems, and that assigning them to certain occupational groups was a secondary process based on the intuitive, unverbalized perception of these attitudes; this accounted for some of the apparent errors in verbal judgment. 

It appears that verbalizing knowledge is different from knowing about something. Special training in any field is directed toward consciously increased selectivity in scanning configurations, and refinement of verbalization, but these are secondary processes, and the primary perceptive processes take place below the level of consciousness. When the purposeful scanning becomes integrated with the total personality through training and experience, it too takes place below the threshold of consciousness, and this integration leads to greater confidence in stating the results of the scanning. The scanning itself gradually becomes integrative rather than additive. Much of the scanning is conditioned by early experiences, so that different individuals integrate different constellations of qualities and potentialities in observing the people they meet. Later experiences, however, such as professional training, can bring about a similar integrating ability concerned with new constellations. 

Psychologically speaking, an amateur in any field becomes a professional when his scanning processes sink below the level of consciousness and function in an integrative rather than an additive fashion. The analogies in the field of motor activity help clarify the point. A beginner dances the rhumba by remembering to put one foot here, then one foot here, and so on, and by this additive process he gets along in an awkward way. After a while he no longer needs to remember, and as a result he dances a smooth, well-integrated rhumba without thinking about it. If he is called upon to explain how he does it, however, he reverts to his former system temporarily. 

\section{Comment}

The reaction to this presentation usually follows certain lines to which the ensuing comments are relevant. 

Since complete records were not kept at the separation center, the number-minded are invited to consider the ideas offered in this report independently of the experience with soldiers. The personality-minded, who might ask what kind of an individual would be interested in such problems, are referred to Sir William Osier, who once made a remark to the effect that a good diagnostician can make a likely preliminary diagnosis by the very way the patient knocks on the door. The idea-minded will be familiar with Bertand Russell's detailed epistemological discussions. Empirically, the fact remains that some individuals, such as professional weight-guessers and experienced clinicians, are better than others at making various kinds of quick judgments about people, depending upon their skill, experience, interest, and receptivity. The writer believes that many people are endowed with much more ability in this regard than they may care to admit. There is no need to bring up questions of ``mental telepathy" on the one hand, nor of ``summation of subliminal cues" on the other, since the problem can be taken care of in a formal way on the basis of configurational judgments without invoking either extrasensory or additive processes. 

\section{Summary}

Every normal individual is able to make diagnoses by inspection. Diagnostic processes, such as those concerning age, are pre verbal and take place below the level of consciousness. The individual may not be aware that he is diagnosing, or if he is, he may not be aware of how he does it. Attempts to explain diagnostic processes are often only justifications, and only occasionally offer criteria which are useful to other diagnosticians with different personalities, at least in the field of psychology. A series is offered to demonstrate the process of diagnosis by inspection in the psychiatric field, and the diagnostic processes are analyzed as far as they became apparent. The development of diagnostic abilities and a frequently occurring reluctance to trust intuitive knowledge are discussed. Preliminary diagnostic processes in experienced clinicians are based on the analysis of configurations below the level of consciousness, and not, as in beginners, on the conscious synthesis of mosaics of observations.

\newpage
\section*{References}

\begin{enumerate}
    \item \label{ref:ref1_mygatt1947} Mygatt, G.: ``Pageant." September 1947.
    \item \label{ref:ref2_darwin1886} Darwin, C.: \textit{The Expression of the Emotions in Man and Animals}. D. Appleton \& Company, New York, 1886.
    \item \label{ref:ref3_krogh1948} Krogh, A.: ``The Language of the Bees." \textit{Scientific American}, August 1948, pp. 18--21.
    \item \label{ref:ref4_wiener1948_time} Wiener, N.: ``Time, Communication, and the Nervous System." \textit{Annals of the New York Academy of Sciences}, L:217, 1948.
    \item \label{ref:ref5_sturtevant1947} Sturtevant, E. H.: \textit{An Introduction to Linguistic Science}, p. 48. Yale University Press, New Haven, 1947.
    \item \label{ref:ref6_deutsch1944} Deutsch, H.: \textit{Psychology of Women}, Vol. 1, p. 136. Grune \& Stratton, New York, 1944.
    \item \label{ref:ref7_schilder1942} Schilder, P.: \textit{Mind}, p. 247. Columbia University Press, New York, 1942.
    \item \label{ref:ref8_brillouin1950} Brillouin, L.: ``Thermodynamics and Information Theory." \textit{American Scientist}, XXXVIII:594--599, 1950.
    \item \label{ref:ref9_bergson1944} Bergson, H.: \textit{Creative Evolution}. Modern Library, New York, 1944.
    \item \label{ref:ref10_jung1946} Jung, C. G.: \textit{Psychological Types}, pp. 567--569. Harcourt, Brace \& Co., New York, 1946.
    \item \label{ref:ref11_reik1948} Reik, T.: \textit{Listening with the Third Ear}. Farrar, Straus \& Co., New York, 1948.
    \item \label{ref:ref12_wild1938} Wild, K. W.: \textit{Intuition}. Cambridge University Press, London, 1938.
    \item \label{ref:ref13_kohler1929} Köhler, W.: \textit{Gestalt Psychology}. Horace Liveright, New York, 1929.
\end{enumerate}

\end{document}