\documentclass[10pt, twoside]{article}
\usepackage[margin=1in]{geometry}
\usepackage{setspace}
\usepackage{titlesec}
\usepackage{fancyhdr}
\usepackage{textcase}
\usepackage{parskip} 
\usepackage{booktabs} 
\usepackage{multicol}
\usepackage{array}
\usepackage{float}
\usepackage{enumitem}

\usepackage[colorlinks=true]{hyperref}

\geometry{
    paperwidth=6in,  % Adjust the width of the paper
    paperheight=9in,   % Adjust the height of the paper
    margin=0.85in,
}

% Line spacing: tighter than double
\setstretch{1.2}

\setlength{\parindent}{1.5em}
\setlength{\parskip}{0pt}

\title{The Nature of Intuition*}
\author{Eric Berne, M.D.} 

\makeatletter
\let\thetitle\@title
\let\theauthor\@author
\makeatother

% Header style
\pagestyle{fancy}
\fancyhf{}
\fancyhead[LO,RE]{\thepage}
\fancyhead[CO]{\scriptsize\MakeTextUppercase{The Nature of Intuition}}
\fancyhead[CE]{\scriptsize\MakeTextUppercase\theauthor}

\renewcommand{\headrulewidth}{0pt}
\renewcommand{\thefootnote}{\fnsymbol{footnote}}

\thispagestyle{empty}

% Section headers
% \titleformat{\section}[block]{\normalsize\bfseries}{}{0em}{}
\titleformat{\section}
  {\normalfont\centering\small}
  {\Roman{section}.}
  {0.5em}
  {\MakeUppercase}

\titleformat{\subsection}
  {\normalfont\itshape\normalsize} % font: normal baseline + italic + small
  {\alph{subsection}.}        % numbering: A, B, C...
  {0.5em}                     % spacing between number and title
  {} 

% Custom maketitle with slightly bold, uppercase title
\renewcommand{\maketitle}{
  \begin{center}
    {\fontseries{bx}\normalsize\MakeTextUppercase\thetitle \par}
    {\footnotesize\MakeTextUppercase{by \theauthor}}
  \end{center}
}

\newcommand{\sref}[1]{\textsuperscript{\ref{#1}}}

\begin{document}

\maketitle

\footnotetext{* Based on a paper read before the annual joint meeting of the San Francisco and Los Angeles PsYchoanalytic Societies on October 18, 1947.}

\section{Introduction}

It appears that under favorable conditions most, if not all, human beings, particularly specialists in various fields of science and commerce, can make judgments about the everyday matters of their concern by the use of functions whose processes are not ordinarily verbalized. In practice, judgments of reality are probably made through the integration of a series of types of cognitive processes (Bergson 1944\sref{ref:ref1_bergson1944}). It is possible, for purposes of investigation, to separate this possibly continuous series into artificial segments. In different situations, different segments of the series would make the major contribution to the verbalized perception. 

First, judgments can be made by means of logic and actively directed, verbalized perception: e. g., the clinical diagnosis of schizophrenia as made by a group of medical students. This is a conscious process. 

Second, they can be made by means of unverbalized processes and observations based on previously formulated knowledge which has become integrated with the personality through long usage, and therefore functions below the level of consciousness; very much as the act of tying a shoelace must be learned by consciously thought-out steps, but later is performed ``automatically" because the kinesthetic image has become integrated with the personality to such an extent that conscious awareness of how it is done is no longer required. This may be called a ``secondarily subconscious" process. ("repression proper" or ``after-expulsion" - Freud) The diagnosis of schizophrenia as made by a specialist may be based on such processes and sensory clues, which, having been verbalized at one time, are perceived and integrated at a later period below the threshold of consciousness (subconsciously**\footnotetext{** This is a legitimate use of a word many people prefer to avoid. Here it is comfortable since it includes both pre-conscious and unconscious.}). He may make the diagnosis on sight and perhaps only later verbalize his mental processes for his students. The group of students makes the diagnosis by a conscious synthetic process, while the specialist may make it by an intuitive process which he is afterward able to analyze. 

Third, judgments can be made with the help of clues whose formulation has not yet become and may never become conscious, but which nevertheless are based on sense impressions, including smell. ("primal repression" - Freud) This may be called a ``primarily subconscious" process. The professional weight-guesser makes continual use of this intuitive process. His uncannily accurate guesses are based on sensory data which he cannot adequately analyze or verbalize, just as the painter may uncannily convey the age and vicissitudes of his subject through his non-verbal medium. The present study is chiefly concerned with this type of intuition, and the writer's observations show that such intuitions are synthesized from discrete sensory elements ("subliminal perceptions") whose perception and synthesis both take place below the threshold of consciousness. Analogous perceptions are spoken of by Freud as forming part of the ``day's residue" in dreams. 

Fourth, they may be made in ways which are quite unexplainable by what we know at present concerning sense-perceptions. 

The first method is evidently a function of the conscious perceptive system. The second and third methods are probably functions of preconscious systems, since they can be brought into conscious analysis relatively easily, and because of their analogy to the use of preconscious material in dreams. The indications are that the fourth method is a function of unconscious systems (Eisenbud\sref{ref:ref2_eisenbud1946}). 

It is probable that judgments, about other people at any rate, are in most cases, if not all, a function of the whole epistemological series and rarely, if ever, the outcome of only one of these artificial segments of it. Since this discussion is mainly concerned with the third method, however, that which has been termed ``primarily subconscious," it should be noted that various authors have expressed valuable opinions which can assist in differentiating the use of such processes in making judgments about people. 

There is a class of ``hunches" in everyday life and of judgments in clinical practice which appear to lack a specific basis in conscious or preconscious experience, and which probably belong here. Such are the experiences of ``listening with the third ear" described by T. Reik\sref{ref:ref3_reik1948}. Since we can throw little light on their mechanisms, they will be called simply ``hunches." E. J. Kempf\sref{ref:ref4_kempf1921}, somewhat like Darwin, speaks of understanding emotional states in others by ``reflex imitation through similar brief muscle tensions," and states that by this token `` in a certain sense we think with our muscles." This method of judgment may be called ``intuition through subjective experience" (proprioception). A similar method can be useful clinically in interpreting handwriting, Bender Gestalt tests, and some material in Rorschach tests. This is a little different from the type of intuitive judgment which is based upon extensive clinical experience, such as has been cited in the case of weight-guessers and which will be enlarged upon here in later clinical material. In Jung's terminology\sref{ref:ref5_jung1946}, intuitions of the latter type are ``objective" and ``concrete." Such intuitions may be termed ``intuition through objective experience." 

Many authors have described other types of ``intuition" under that name\sref{ref:ref6_poincare1948} or something similar, such as ``inspiration"\sref{ref:ref7_kris1939}, ``insight,"\sref{ref:ref8_hutchinson1939} etc. On the other hand, many of the magnificent edifices of the philosophers, such as Kant, Descartes, and Locke, use the concept of intuition as one of their building blocks. If we aspire here only to consider what is commonly called ``clinical intuition," we avoid the dangers run by those who try to scale the walls of philosophy. The philosophical aspects have been discussed by K. W. Wild\sref{ref:ref9_wild1938}.

For the present purpose it is only necessary to define intuition sufficiently to separate it from its nearest neighbors. A pragmatic definition, based on clinical experience, may be stated as follows: 

Intuition is knowledge based on experience and acquired through sensory contact with the subject, without the ``intuiter" being able to formulate to himself or others exactly how he came to his conclusions. Or in psychological terminology, it is knowledge based on experience and acquired by means of pre-verbal unconscious or preconscious functions through sensory coniact with the subject. This approximates the definition of Jung\sref{ref:ref5_jung1946}, who says that intuition ``is that psychological function which transmits perceptions in an unconscious way." It is even something like the dictionary definition: ``the quick perception of truth without conscious attention or reasoning." (Funk \& Wagnalls.) 

This concept of clinical intuition implies that the individual can know something without knowing how he knows it. ("That distant cow is sick.") If he can correctly formulate the grounds for his conclusions, we say that they are based on logical thought ("This cow is sick because...") and actively directed observation ("This is obviously the sick one."). If his conclusion seems to be based on something other than direct or indirect sensory contact with the subject ("Somewhere a cow is sick"), then we cannot help but be reminded of what J. B. Rhine calls ``extra-sensory perception"\sref{ref:ref10_rhine1937}.

After careful consideration, it will be found that an interesting corollary must be added to this definition. Not only is the individual unaware of how he knows something; he may not even know what it is that he knows, but behaves or reacts in a specific way as if (\textit{als ob}) his actions or reactions were based on something that he knew. 

The problem of intuition is related to a general question which may be stated thus: 

From what data do human beings form their judgments of reality? 

(By \textit{judgment} is meant an image of reality which affects behavior and feelings toward reality. An \textit{image} is formed by integrating sensory and other impressions with each other and with inner tensions based on present needs and past experiences. By \textit{reality} is meant the potentialities for interaction of all the energy systems in the universe; this implies the past.) 

Regarding the special matter of concern here, the ``primarily subconscious" material which forms the basis for judgments about external reality, Reik\sref{ref:ref3_reik1948} has made some formulations with which the present conclusions, based on clinical experimental material, are in agreement. This is all the more impressive since the latter were arrived at independently after the pertinent observations had been made, during: (1) Attempts to intuit single specific factors in a series of several thousand cases. (2) Attempts to intuit many different factors about single individuals. 

Curiously enough, among philosophers, the man whose ideas come closest to these conclusions is one of the most ancient. It was Aristotle who described what has been called ``intuitive induction" as being based on the ability of the organism, first to experience sense-perceptions; at a higher level of organization, to retain sense-perceptions; and at a still higher level, to systematize such memories. ``We conclude that these states of knowledge are neither innate in a determinate form, nor developed from other higher states of knowledge, but from sense-perception. It is like a rout in battle stopped by first one man making a stand and then another, until the original formation has been restored..."\sref{ref:ref11_aristotle1934} It is also apparent how closely Aristotle's remarks are related to the discussion of the similarities between neurophysiological phenomena and the functioning of calculating machines which is part of the subject of cybernetics according to N. Wiener\sref{ref:ref12_wiener1948}.

The clinical material has a special bearing on one aspect of this question: Namely, from what data other than rational conclusions and consciously perceived sense impressions do human beings form judgments about external reality? ("Consciously perceived sense impressions" are those which can be readily verbalized, in contrast to ``subconscious perceptions"\sref{ref:ref13_hinsie1945} and the ``subliminal cues" of modern psychology.)

\section{Clinical Material}

\subsection{Observations of Single Specific Factors in Large Numbers of Individuals}
These observations were made at an Army Separation Center in the latter part of 1945. One part of the processing consisted of a medical examination carried out in assembly-line fashion. Each soldier went down a line of booths, and in each booth certain organ systems were examined and the results noted in the appropriate places on a printed form. The writer was in a booth at the end of the line. The time available for the ``psychiatric examination" varied on different days from 40 to 90 seconds. About 25,000 soldiers came down the line in less than 4 months. Several studies were made during this period, and about 10,000 cases were available for the study of the intuitive process.

The study was not formulated premeditatively. The writer became interested gradually in the nature of the process which, with practice, enabled him to detect and distinguish accurately some categories of human beings after 10 or 20 seconds of inspection.

The men all wore the same garments: a maroon bathrobe and a pair of cloth slippers. The examiner sat behind a desk, facing the door of the booth. After a soldier was ``examined," the appropriate blank was filled in on the form, and the next candidate was summoned by calling ``Next!" As one soldier left, the next shuffled in and, without instruction, walked toward a chair beside the desk to the right of the examiner and sat down. Some soldiers kept their papers in their hands, and some handed them to the examiner. These forms were looked at after the interview was ended. It was not necessary to know the names of the soldiers.

The ``examination" consisted of 2 stock questions, asked after a few moments of inspection: ``Are you nervous?" and ``Have you ever been to a psychiatrist?" At first, that was all, unless there were special indications. During this preliminary period, an attempt was made to predict, from silent observation of the soldier, how each man would answer the 2 stock questions in that particular situation. It was found that this could be done with surprising accuracy. The question then arose as to how these predictions were made, since this was not immediately apparent. After careful study, the question -- ``How are such intuitive judgments made, and upon what are they based?" -- was partly answered for the factors concerned.

It seemed evident, however, that the formulation was not completely successful, for the percentage of correct predictions remained higher when the intuitive process was allowed to function without conscious interference than when judgments were attempted on the basis of deliberate use of the criteria that had been verbalized. The conclusion drawn was that not all of the criteria used in the intuitive process had been formulated. A discussion of the nature of these particular criteria, and their psychodynamic and psychiatric implications, will not be undertaken here.

When it was thus found almost by accident that the intuitive process could be studied in that particular situation, a more formal experiment was undertaken. An attempt was made to guess, by observing the soldier for a few seconds, what each man’s occupation had been in civilian life, and then to formulate the data upon which the guesses were based. During this experiment, intuitions regarding the answers to the routine questions about nervousness were forthcoming as well, with practically no additional effort, and continued to be useful in picking out false-negative replies. This means that two fields of intuition were active at the same time.

Fortunately, then, the experiment did not interfere with the duty of making the best possible psychiatric evaluation of each man in the time available; and, I was informed later, it added interest and spirit to the routinized experience of each man’s examination. Since the center was not set up for experimental psychology, no control of the results was possible other than through the individual soldiers who went through the experience, except occasionally during a lax period, when some medical officer from a neighboring booth would drop in.

During the examination, the soldiers were under emotional tension related to a uniform goal-striving -- namely, their desire to get out of the army as soon as possible, for they believed that the doctors could frustrate this desire. This tension was particularly high when they entered the psychiatrist’s booth, because of the especially imponderable (in their minds) nature of his function. The interview was an emotionally charged ``examination" crisis, and not an artificial laboratory situation.

This was emphasized in that environment by the fact that the soldiers were unclothed and were enlisted men, while the examiner was fully clothed and an officer. Upon becoming participants in this situation, each was met by a neutral but unswerving gaze, and by silence and obvious ``observation," in a fashion which only a few, if any, could have experienced before. Thus, for most of them it was an imponderable, anxiety-laden, and new situation.

Since written protocols were not regularly kept, numerical data are available for only a small sampling of the study. On 17 different days, the guesses or lack of them were recorded for ``unselected" segments of the line-up, comprising in all 391 cases. In 84 of these cases, no attempt was made at guessing the occupation, as no clear impression was obtained by inspection. In the remaining 307 cases, guesses were made and recorded. Of these guesses, 168, or 55\%, were correct, and 139, or 45\%, were incorrect.

On other days, when intrinsic distractions (as opposed to extraneous stimuli) were operating -- as on the day when the separation center was deactivated -- only about 1/4 as many correct guesses were made as on days when intuition was operating free from relevant emotional interferences: e.g., 14\% correct guesses as compared with 55\%. A similar fall in accuracy usually occurred as fatigue set in, if more than 50 guesses in succession were attempted. It was noted that there was a ``learning period" of about 2 weeks when the study began, during which the reliability of the intuitive process gradually increased, after which no further significant increase was demonstrable.

Records on this subject were spread over a period of 47 days, interspersed with other studies. The following is the first half of a statistically average record, presented verbatim. (The special notes, including those referring to ``eye sign," will be discussed later.)

Throughout the study, as exemplified, continual attempts were made to verbalize the grounds for the judgments. Whenever a criterion was satisfactorily verbalized, it was tested on several hundred cases. It was found again, as in the case of diagnosing ``neurotic behavior" in the preliminary period, that reliance on such formulated criteria yielded less reliable results than intuition. Each time a new criterion was added to the formulation, the percentage of hits went up, but never reached the level attained through the use of intuition during ``intuitive periods.”

The occupations most closely studied were ``farmers" and ``mechanics." These were the 2 groups which the examiner became most adept at diagnosing. From the series of 307 recorded guesses, 58 out of 79 guesses of ``farmer," or 73\%, were correct, while 14 actual farmers, or 20\% of their total, were wrongly assigned. Similarly, 17 out of 32 guesses of ``mechanic," or 53\%, were correct, with 10 actual mechanics, or 37\% of their total, wrongly assigned. During the whole run of the experiment, recorded and unrecorded, which included an estimated 2,000 cases -- about 50 cases a day for about 6 weeks -- the percentages of correctly recognized farmers and mechanics were high. The study of intuition in connection with these 2 occupational groups revealed some of the properties of the process. The following formulations gradually emerged, as the basis for each separate judgment was studied.

\begin{enumerate}[leftmargin=0pt,itemindent=3em,labelsep=0.5em]
    \item Certain men, when they met the examiner’s neutral gaze, shifted their eyes to the left and stared out of the window. The examiner came to refer to this, in his own mind, as the ``farmer’s eye sign." It was felt, however, that this was not the whole story and that something was being missed -- that the intuitive results were based on something more that was being observed, but was not captured in this verbalization.
    \item This uneasy feeling was confirmed by the fact that when intuition was suspended and this ``eye sign" criterion was consciously applied, there were many more errors in determination. Study of these errors led to a refinement and reformulation of the criterion. The true ``farmer’s eye sign," which was, with few exceptions, peculiar to farmers in the given situation, was found (a) to occur only in individuals whose faces froze after a few seconds into a stolid expression, and (b) to consist of a special type of gaze shift to the left -- namely, a slow and expressionless one. A rapid shift, or an alert expression during the shift, was not often seen in members of this occupational group.
    This is noted in Case 21, where the erroneous guess of ``farmer" was made. The man said that his regular occupation was truck driver, and it was then noted: ``Eye sign too fast for farmer." In Case 15, ``oil fields" was the guess, but the man said he was a farmer. He was then retested for the farmer’s eye sign, and it was found to be positive. In the next case, No. 16, the guess was ``raised on farm, worked in factory later," and it was noted that the farmer’s eye sign was present in a modified form. The nature of this frequently occurring type of modification was not successfully verbalized.
    \item Since the refinement in active observation of the farmer’s eye sign still resulted in a lower level of correct hits than did the use of ``intuition," other objectively definable factors were sought. The examiner began to take conscious note of the complexion, which had not been done before. This proved unreliable by itself, but if thoughtfully correlated with the eye sign, it helped in a good many cases and decreased the negative errors -- that is, guessing something other than ``farmer" for a farmer. It did not, however, decrease the positive errors -- that is, guessing ``farmer" in cases of other occupations (Case 5). (This result has implications which are not sufficiently important or well-founded, given the evidence at hand, to warrant discussion.)
    Since the examiner did not consciously direct attention to the hands unless otherwise baffled, as in Case 9, the extent of their diagnostic influence in this situation is unknown (F. Ronchese\sref{ref:ref14_ronchese1945}).
\end{enumerate}

\section*{Protocol No. 1}
{
\renewcommand{\arraystretch}{1.3}
\noindent\centering
\scriptsize
\begin{tabular}{
  l
  p{0.22\textwidth}
  p{0.25\textwidth}
  p{0.30\textwidth}
}
\toprule\toprule
 & \multicolumn{1}{c}{\textbf{Guess}}
 & \multicolumn{1}{c}{\textbf{Inquiry}}
 & \multicolumn{1}{c}{\textbf{Notes}} \\
\midrule
1  & Truck or factory & Truck or factory & (Short, alert, stocky) \\
2  & Lawyer or small store-keeper & Lawyer &  \\
3  & Farmer & Farmer & (Eye sign present) \\
4  & Machinist or truck driver & Truck driver &  \\
5  & Farmer & Milkman & (Had suitable complexion but not the eye sign, and I doubted it) \\
6  & No guess made & Ranch and bull stud man &  \\
7  & No guess made & Auto body, welding, etc. &  \\
8  & Something to do with automobiles & Truck driver &  \\
9  & Truck driver & Truck driver & (Something about the mouth and the way the hands are held; or wrists?) \\
10 & Farmer & Farmer & (Eye sign) \\
11 & Mechanic & Mechanic and carpenter & (i.e., ``uses hands") \\
12 & Sales or office & Farm or factory & (Uncertain soft voice; anxiety state, moderate) \\
13 & Contractor & School teacher & (Handles, i.e., bosses people) \\
14 & No guess made & Steel mill &  \\
15 & Oil fields & Farm & (Retested for eye sign and was positive) \\
16 & Raised on farm, worked in factory later & Raised on farm, worked in factory later & (Eye sign modified) \\
17 & Raised on farm, worked in a big city & Raised on farm, worked in a big city as plumber and mechanic &  \\
18 & Truck driver & Truck driver in army, in civilian life was sexton in cemetery &  \\
19 & I don't know, probably a mechanic & Logger. Truck driver in army &  \\
20 & No guess & Truck driver &  \\
21 & Farmer & Truck driver, small town & (Eye sign too fast for farmer) \\
\bottomrule\bottomrule
\end{tabular}
}

In the case of mechanics, the verbalization which gradually took form was as follows: 

Certain men, when they met the examiner's gaze, looked straight into his eyes with an expression of lively curiosity, but without challenge. (Because of ``challenge," guessing by eye sign was unsuccessful with officers, and the sign was found to be applicable only to enlisted men in this particular situation\footnotetext{* It was a long time after the event before it occurred to me that ``challenge" itself constituted an ``officer's eye sign" in the given situation.}.*) This group generally proved to be mechanics. Where a positive ``mechanic's eye sign" was present but the man said he was not a mechanic, he be, longed in many cases to an allied trade, such as radio technician. This observation has its own significance, which can be discussed later. 

Men of other occupational groups manifested a variety of eye movements which did not seem to be specifically correlated with their occupations. 

The diagnosis of ``truckdriver `` was correct in 22 out of 36 recorded cases, or 61\%. It was overlooked 11 times in the 307 recorded cases. Attempts to verbalize in connection with this occupation were made (as in Case 9), but they were not successful. The same applies to construction workers, who were frequently picked out successfully. It was noted that these were often of mesomorphic, or combined athletic-pyknic, physique, but no further clues could be verbalized. 

Some of the individual guesses were interesting, in that in a few cases, factors other than occupation were intuited. A passive attitude of mind was maintained, oriented toward ``occupation," but it happened occasionally that a man gave such a strong impression concerning some other factor, that ``occupation" was heavily overshadowed. This frequently happened in the case of New Yorkers, who, silent in their bathrobes, sometimes gave such a strong impression of being, above all, New Yorkers, that other intuitions seemed to be put in umbrage. There was one professional gambler among the 25,000 men, and he was picked out successfully. Salesmen were picked out with considerable regularity, but only after they had talked, and the notes 'in such cases are revealing; for example: ``Deep voice, good animation -- a talker." ``Good talker -- also they say more than the others, instead of just 'yes' or 'no.' `` The verbalized criterion in the case of salesmen was: `` If he appears to 'love' his voice, he is most likely a salesman. His voice is important to him as an instrument for dealing with reality." This verbalization has interesting psychodynamic implications. 

This observation in the case of salesmen, and further consideration of the occurrence of ``eye signs" in farmers and mechanics, gradually led to a new and even startling line of thought which was helpful in the attempt to understand the intuitive process. It was found eventually that in effect it was not occupations at all which were being judged, but attitudes toward reality problems. It appeared that the positive farmer's eye sign did not mean ``farmer," so much as ``one who waits stolidly in the face of an imponderable situation"; while the positive mechanic's eye sign signified not ``mechanic," but ``one who is curious to know what will happen next and how things will work out." This accounted for the nature of some of the errors, as in guessing ``mechanic" in the case of a radio technician. The question of what heritage, which experiences, and what instinctual constellations conditioned these eye signs, is beyond the present scope.

\subsection{Observations of Numerous Factors About Single Individuals}


One may now turn from intuitions based on the manner in which
the individual met a novel and anxiety-laden present' reality situation to those which had another basis and dealt with other aspects of the individual's personality. From a collection of cases,
a few may be selected which are particularly pertinent to the present discussion. These reveal to what extent the subject can communicate information concerning elements with which the intuiter has no direct contact.

\section*{Protocol No. 2}


During tours of night duty in various army hospitals, the writer adopted the custom of passing time with patients on the wards whenever opportunity offered. One evening, upon entering an unfamiliar ward, I found a patient who was unknown to me sitting in the office. Knowing that he should not have been there, he rose with an apology; but I felt that he was an interesting and intelligent individual and suggested that he remain. After this brief exchange of politeness and a few moments of contemplation I ventured to guess, correctly, the city of his birth and the age at which he had left home. The conversation then proceeded as follows:

\subsection*{Case 1}

\begin{itemize}[leftmargin=0pt,itemindent=3em,labelsep=0.8em]
    \item[Q.] I believe your mother ``disappointed" you.
    \item[A.] Oh, no, sir. I love my mother very much.
    
    \item[Q.] Where is she now?
    \item[A.] She's at home. She's not well.
    
    \item[Q.] How long has she been ill?
    \item[A.] Most of her life. I've been taking care of her since I was a young fellow.
    
    \item[Q.] What's her trouble?
    \item[A.] She's always been nervous. A semi-invalid.
    
    \item[Q.] Then in that sense she ``disappointed'' you, don't you think?
    She had to take emotional support from you rather than give it to you, from your earliest years.
    
    \item[A.] Yes, sir, that's correct, all right.\\
    
\end{itemize}
    
At this point another man who was a stranger to me entered the office, and was invited to sit down. He sat on the floor with his back against the wall and said nothing, but listened with great interest.\\
    
\begin{itemize}[leftmargin=0pt,itemindent=3em,labelsep=0.8em]
    
    \item[Q.] (To the first man.) You give me the impression that your father was ineffective from the time you were about nine.
    \item[A.] He was a drunkard. I believe about the time I was nine or ten he began to drink more heavily.
\end{itemize}

\subsection*{Case 2}

After listening to a few more such exchanges, the second man
requested to be told something about himself.\\

\begin{itemize}[leftmargin=0pt,itemindent=3em,labelsep=0.8em]
    
    \item[Q.] Well, I think your father was very strict with you. You had
    to help him on the farm. You never went fishing or hunting with
    him. You had to go on your own, with a bunch of rather tough
    fellows.
    \item[A.] That's right.
    
    \item[Q.] He began to scare you badly when you were about seven
    years old.
    \item[A.] Well, my mother died when I was six, if that had anything
    to do with it.
    
    \item[Q.] Were you pretty close to her?
    \item[A.] I was.
    
    \item[Q.] So her death left you more or less at the mercy of your
    father?
    \item[A.] I guess it did.
    
    \item[Q.] You make your wife angry.
    \item[A.] I guess I did. We're divorced.\\
    
\end{itemize}

This took me by surprise. After a moment we proceeded:\\

\begin{itemize}[leftmargin=0pt,itemindent=3em,labelsep=0.8em]
    
    \item[Q.] She was about sixteen and a half when you married her.
    \item[A.] That's right.
    
    \item[Q.] And you were about nineteen and a half when you married
    her?
    \item[A.] That's right.
    
    \item[Q.] Is it right within six months?
    \item[A.] (Pause.) They're both right within two months.
    
    \item[Q.] Well, fellows, that's as far as I can go.
    \item[A.] Will you try to guess my age?
    
    \item[Q.] I don't think I'm in the groove for guessing ages tonight.
    I think I'm through.
    \item[A.] Well, try, sir.
    
    \item[Q.] I don't think I'll get this, but I'll try. You were 24 in September.
    \item[A.] I was 30 in October.
    
    \item[Q.] Well, there you are.\\
    
\end{itemize}


About a week later, these men, with their consent, appeared in a clinic designed to demonstrate how the early emotional adventures of the individual leave their marks not only on his later personality, but also on his muscular set, particularly about the face. On that occasion I had an opportunity to learn their names and read their case histories. Some time later one of the men was encountered in civilian life, at which meeting some of the intuitive deductions were reconfirmed. Away from the artificial situations of army life, we are still good friends.

\section*{Protocol No. 3}
\subsection*{Case 1}

At the request of two psychiatric colleagues in the army, I interviewed, in their presence, a new arrival on their ward to ascertain whether the delicate intuitive process could function under conditions of controlled observation. I found that after asking the subject a few ``irrelevant" questions, in order to gain an impression of the dynamics of his voice and facial muscles, it was possible to make some conjectures about his early relationships with his parents, his work history, the destiny of his later relationships, and other factors. It was correctly surmised, for example, that he changed jobs frequently because of misunderstandings with his employers, but that he had finally settled into a position in which no one supervised him, and had managed to hold this job much longer than any of the others. The important point, however, is not that some of the guesses were correct, but that none of them was incorrect. This incident gave a distinct impression that intuition is sometimes able to function when ``put on the spot.”


\subsection*{Case 2}

All three of us were interested in pursuing the matter further, and an opportunity presented itself with the arrival of a new patient from another service for psychiatric consultation. The senior psychiatrist made some of the usual anamnestic inquiries of the 27-year-old bachelor and then asked for my comments. I ventured to say that, in my opinion, an important precipitating factor in the case was some shock the man had received at the age of 18. (His adolescence had not been investigated during the previous questioning.) The man stated that nothing serious had happened to him during that period of his life. In spite of his statement, I intimated that the strength of my intuition persisted. After further questioning, the senior psychiatrist asked him why he had not yet married, whereupon the patient burst into tears and said: ``I was supposed to be married once; we were all set for a big wedding, and everybody was at church waiting, and she never did show up. That was when I was 18 years old, as the captain said. I didn’t want to tell you about it." When the intuitive mood is strong, it brings with it a feeling of certainty that is difficult to shake off. Just as the man in Case 1, Protocol No. 2, denied that his mother had ``disappointed" him, so this man stated that nothing serious had happened to him at the age of 18; yet further questioning in both cases confirmed the intuitive impression.

\section*{Protocol No. 4}

Many years ago, after some ``irrelevant" conversation with a young woman whose existence I had no reason previously to suspect, I made the following observation.\\

\begin{itemize}[leftmargin=0pt,itemindent=3em,labelsep=0.8em]
    \item[Q.] I have the feeling that you are either the fourth or the seventh of 11 children.
    \item[A.] I am the fourth of 11 children and I have seven brothers.\\
\end{itemize}

This confirmation was apparently more incredible to me than my observation was to the individual in question. Other sources later corroborated her statement. My remark was preceded by a feeling which might be roughly translated as follows : ``If I watch this person closely for a few moments something might occur to me."

\section*{Protocol No. 5}

During the war, while talking to a young woman who was previously unknown to me, I advanced the hypothesis that she had 28 teeth. This hypothesis was based on a sudden ``inspiration" which came to me at that moment without any premeditation. She had not shown her teeth, and my observation, including the number 28, was irrelevant to anything we had discussed, except possibly her sadistic tendencies; nor am I in the habit of enumerating people’s teeth. She herself did not think that my comment was accurate, but we reviewed the situation and found that it was.

\par\bigskip
\centerline{* \qquad * \qquad *}
\bigskip\par

One is sometimes astonished by the accuracy of intuition, as exemplified in the last two protocols and others like them. One would expect that, if one guessed a ``number" in a large series of cases, one would be right in a certain proportion; it is quite another thing to be right almost all the time when in a certain frame of mind. I have observed that when intuition seems strong enough to risk a guess of a ``number," the guess is nearly always accurate. When the ``intuitive mood" is not present, or when intuition is ``put on the spot," guesses of numbers are more likely to be erroneous, as in Case 2, Protocol No. 2. In that case, while intuition was functioning spontaneously, it was possible to guess correctly the age of a man’s unseen wife when he married her; when the mood left, and in the face of a challenge, there was a gross error in guessing the age of the man who stood there in person.

It is true that, unless one actively cultivates intuition at times, such incidents occur only a few times a year. One must control one’s attitude toward such matters. Intuition might be used in practice to make an estimate of a patient’s personality -- an estimate that would become clouded as it was overlaid with clinical material. Usually, however, one would find in the end, when this ``clouded" period had been worked through, that the first intuition was reliable. It is probably detrimental, however, either to record one’s intuitions in ordinary practice or to communicate them to the patient. Such an externalization tends to limit the fluidity of images that is desirable for the best therapeutic results. If one makes a restricted and carefully thought-out communication in this regard to two or three patients for experimental purposes, one easily becomes convinced that such comments are not taken lightly and may have a far-reaching effect on the therapeutic situation. On the other hand, with strangers it is necessary first to establish the proper rapport if one wishes to exercise such privileges; otherwise, difficulties may no doubt arise.

\section{Qualities of the Intuitive Function}

A certain attitude of mind -- the ``intuitive mood" -- is most favorable to the intuitive function. The writer learned little about the ``psychic environment" most conducive to such a mood. Extraneous stimuli need not necessarily be excluded. The soldiers at the separation center were examined in a chilly open booth in a noisy atmosphere of hurry and excitement, and the examiner was able to engage in conversation with colleagues between the brief periods of concentration, each lasting only a few seconds. Notes in the protocols such as ``Room very chilly today" were not followed by any diminution in accuracy. Neither were notes such as ``Up a good part of the night last night," so the relevance of known (extraneous) internal stimuli remains a question requiring further study. On the other hand, the note ``Separation center deactivated today" was followed by a serious loss of intuitive efficiency.

Knowledge of the conditions required to induce the intuitive mood at will would be of great value, but unfortunately no one has yet been able to verbalize these conditions. Such a mood does not resemble the withdrawal from reality that advanced students of Yoga and others are able to attain, since it is possible during intuitive periods to maintain normal relationships with psychiatrists and other individuals. Perhaps a narrowed and concentrated contact with external reality is necessary. The chief requisite seems to be a state of alertness and receptiveness, requiring, however, more intense concentration and more outwardly directed attention than the passively alert state familiar to psychotherapists.

Directed participation of the perceptive ego interfered with intuition. When previously verbalized sensory clues were deliberately sought, the intuitive process was impaired, although it could be immediately revived. This may have some psychodynamic connection with the experience that clinical intuition works poorly with acquaintances of the intuiter and functions best with complete strangers. Deutsch\sref{ref:ref15_deutsch1944} remarks that intuition ``will naturally depend on one’s sympathy and love for and spiritual affinity with the other person," but I have found that, in general, previous acquaintance with the subject is an obstacle rather than an asset. In special cases, however, where the ``clouded period" referred to above has been successfully worked through in either a professional or personal relationship, her statement takes on its fuller meaning. The problem of resistance in this connection remains to be clarified.* \footnotetext{* It was resistance and counter-transference that at first blinded me to the fact that ``challenge" was a diagnostic sign for officers in the given situation, and made me feel instead that it was an obstacle. Detailed analysis of this interesting insight is beyond the present scope.
}(Pederson-Krag\sref{ref:ref16_pedersonkrag1947}). Similar factors probably tend to hamper intuition when the intuiter is ``put on the spot." He needs a mechanism for dealing with any anxiety aroused by such a situation, or intuition is likely to fail, even if the subject is a stranger. 

With practice, the intuitive mood can be attained more easily. Unless one is in good form, it is difficult to become intuitive at will. Many psychiatrists and psychoanalysts successfully use intuition day after day in active practice, but sometimes after a vacation period find their intuition ``rusty." Specialists in other professions who work partly by intuition often find that after a holiday, although they may return with a fresh mind and viewpoint, their intuition is less effective until they are back in the swing of regular work. A similar contrast exists between the regular daily exercise of intuition at the separation center and the sporadic occurrence of the intuitive mood when it was not in daily use. 

The intuitive function is fatigable: after about fifty successive guesses at the separation center, the percentage of correct guesses fell off markedly. Despite the subjectively observed inactivity of some ego functions, intuition is fatiguing. The type of fatigue may be compared to that felt after difficult mental strain, such as a hard game of chess.

There was considerable evidence that accuracy improved with accumulated experience in each field, though the possibility of a plateau effect once a certain level is reached cannot be eliminated. The case of the woman with twenty-eight teeth, as well as other cases, raises the question of whether extensive prior experience in a given field is always a prerequisite for intuitive accuracy. It was also interesting to note that accuracy was not diminished when judgments were sought in two different fields at the same time (for example, ``degree of neuroticism" and ``occupational group”), suggesting that intuitions do not interfere with one another.

Some of these conditions are reminiscent of those mentioned by Rhine in connection with what he calls the ``extrasensory perception" function. The conditions outlined here may be summarized as follows: 

``The intuitive mood is enhanced by an attitude of alertness and receptiveness without actively directed participation of the perceptive ego. It is attained more easily with practice; it is both fatigable and fatiguing. Intuitions in different fields do not seem to interfere with one another. Intuitions are not all dependent upon extensive past experience in the given field. Extraneous physical stimuli, both external and internal, appear to be largely irrelevant."

Some self-observation during the intuitive process yielded a kind of introspective formula which can be stated as follows : ``Things are being 'automatically' arranged just below the level of consciousness; 'subconsciously perceived' factors are being sorted out, fall 'automatically' into place, and are integrated into the final impression, which is at length verbalized with some uncertainty." Again one is reminded of the recent cybernetic formulations. 

The more prolonged the gaze, the greater the amount of the material which seemed to go through the process, and the greater the number of the impressions which could be verbalized. When the perceptive ego was not directed, the activity of some other function could be ``felt," and the fatigue of this latter function could be sensed if an attempt was made to continue too long.



\section{What Is Intuited?}

We have evidence that an intuition consists of two processes: ``subconscious perception" and conscious verbalization. At the separation center, the conscious verbalizations were at first naively accepted as direct formulations of the intuitions themselves. It was thought that the intuitive function was actually perceiving ``occupational group." Later it became apparent that what the intuitive function really perceived was an ``attitude toward an imponderable reality situation." The intuiter’s ego then translated these perceptions into a judgment concerning occupational group.

With the men on the ward (Protocol No. 2), a similar process took place. For example, one verbalization was: ``She was about sixteen and a half when you married her," and it was assumed that this was what had been intuited. Actually, in retrospect, the preconscious material seemed to have run more as follows: ``This is a man who lacked feminine influence in later childhood and wanted to get away from his father. Such a man, as I see before me, married young and impulsively, choosing a wife on the basis of certain needs and anxieties of the moment. In this type of case she would be a few years younger than himself and as ‘lost’ as himself. Therefore, he married a girl who was ready to marry at the age of sixteen and a half.”

Later, the corollary was formulated on the basis of the intuition, ``The situation came to a head in late adolescence," and was verbalized as: ``He married when he was nineteen and a half years old." (In this case the actual ages have been changed slightly for reasons of discretion.)

We are thus led to believe that there are at least two types of factors that may be intuited: attitudes toward reality and instinctual vicissitudes -- or, more succinctly, ego attitudes and id attitudes. These may then be verbalized into guesses, for example, of occupational group and object choice, respectively.

There seemed to be specific clues related to each of these factors. The subject’s attitude toward an imponderable reality situation was usually gauged primarily from clues supplied by the eyes and the periocular muscles. Impressions concerning the instincts and their vicissitudes, by contrast, were largely based on ``subconscious observation" of the muscles of the lower face, especially those around the mouth. Head posture and mannerisms based on the tonus of the neck muscles can also be indicators in this respect.

One might say that in these situations the eyes were principally instruments of the ego, while the mouth and neck were more expressive of the functions of the id.

\section{Discussion}

The material presented here has offered an opportunity to discuss, supported by a number of clinical examples, ideas which have been the subject of speculation for many centuries. In attempting to place these findings in a broader frame of reference one arrives at viewpoints similar to those of Bergson\sref{ref:ref1_bergson1944} and Reiks\sref{ref:ref3_reik1948}. Standing on the small island of the intellect, many are trying to understand the sea of life; at most we can understand only the flotsam and jetsam, the flora and fauna which are cast upon the shores. Taking a verbal or mechanical microscope to what we find will help but little to know what lies beyond the horizon or in the depths. For this we must swim or dive, even if the prospect dismays us at first. 

To understand intuition, it seems necessary to avoid the belief that in order to know something the individual must be able to put into words what he knows and how he knows it. This belief, still common since Freud, is the result of what appears to be an overdevelopment of reality testing which tempts some who are interested in psychology to think too far away from nature and the world of natural happenings. Dogs know things, and so do bees (von Frisch, Lubbock) and even \textit{stentor} (Jennings). True knowledge is to know how to act rather than to know words. If a certain man looks out of the window in a certain way, we may know how to behave toward that man and what to expect from him. If another man looks at us with lively curiosity, we may know how to behave toward and what to expect from him. To put what we know about these men into words is quite another matter. The relationship of such matters to intragroup reactions (i. e., through what mediums other than words do people provoke and communicate with each other) and to the ``undirected function" of the central nervous system (Federn\sref{ref:ref17_federn1938}) remains to be clarified.

In attempting to ``isolate" operations, particularly operations of the human mind, one is reminded that the concept, ``isolation of an operation," is itself a creation of the human mind. Since the mind is in such cases attempting to think about itself with itself as an instrument, a difficulty arises allied to the kind of difficulty which in logic is typified by Epimenides (B. Russell’s discussions of ``paradoxes”). Just as some statements about propositions must be analyzed differently from other classes of propositions, so mentation about mental phenomena may be considered differently from mentation about other natural phenomena. The future of psychology may lie in the paradoxes rather than in the body of logic. (the modern methodological approaches of Einstein, H. Weyl, Korzybski, N. Wiener, et al.)

In a previous publication\sref{ref:ref18_berne1947} in which some of the material studied in this paper is mentioned briefly, I summarized the problem along the following lines: In subduing the forces of the id, man often imprisons much that could be useful and beneficial to the individual. Many people could cultivate intuitive faculties without endangering the rest of their personalities and their necessary testing of reality.* \footnotetext{* On the contrary, my initial failure to recognize “challenge” as a diagnostic sign for officers was evidence of involvement with my own questionable anxieties of the moment; the subsequent recognition of the intuitive value of this phenomenon represented freedom and insight and improved reality testing.
}

Freud left confident enough to imply that there is no need to be alarmed by proposals of this nature\sref{ref:ref19_freud1933}. One might even go so far as to agree that in everyday life people learn more, and more truly, through intuition than they do through verbalized observations and logic. We are tempted to be proud of verbalizations, but it is possible that in many of our most important judgments the small and fragile voice of intuition is a more reliable guide.

Wittels has outlined the weaknesses of intuition\sref{ref:ref20_wittels1945}: ``(1) one has to be endowed with it, (2) it may lead us astray, (3) soon a definite limit is reached beyond which there is no further progress without scientific method. I have never met a man who could equal Freud in intuition, i.e., of inexplicable immediate psychological insight. But he also had scientific self-control which--with a few exceptions--did not trust his unproved visions." To which an optimistic man might reply: (1) that he believes everyone is endowed with intuition and needs only to get at it; (2) that it will not lead us astray if we can free it from destructive involvement with neurotic constellations and anxieties; and (3)\sref{ref:ref6_poincare1948} that there is a time for scientific method and a time for intuition--the one brings with it more certainty, the other offers more possibilities; the two together are the only basis for creative thinking.

\section*{Conclusions}

\begin{enumerate}
    \item An intuitive function exists in the human mind.
    \item Under proper conditions, this function can be studied empirically.
    \item The intuitive function is part of a series of perceptive processes which work above and below the level of consciousness in an apparently integrated fashion, with shifting emphasis according to special conditions.
    \item The clinical intuitions studied were found in most cases to be based at least partly on preconscious, sensory observation of the subject.
    \item What is intuited is different from what the ``intuiter" verbalizes as his intuition.
    \item The dynamics of the eyes and the periocular muscles express reality attitudes. The dynamics of the lower facial and neck muscles are more indicative of instinctual vicissitudes.
    \item Intuitive faculties may be more important than is often admitted in influencing judgments about reality in everyday life.
    \item The intuitive function is useful and worth cultivating.
\end{enumerate}

\newpage
\section*{References}

\begin{enumerate}
    \item \label{ref:ref1_bergson1944} Bergson, H.: \textit{Creative Evolution}. Modern Library, New York, 1944.
    \item \label{ref:ref2_eisenbud1946} Eisenbud, J.: Telepathy and Problems of Psychoanalysis." \textit{Psychoanalytic Quarterly}, XV:32--87, 1946.
    \item \label{ref:ref3_reik1948} Reik, T.: \textit{Listening with the Third Ear}. Farrar, Straus \& Co., New York, 1948.
    \item \label{ref:ref4_kempf1921} Kempf, E. J.: \textit{The Autonomic Functions and the Personality}, p. 23. Nervous \& Mental Disease Pub. Co., New York, 1921.
    \item \label{ref:ref5_jung1946} Jung, C. G.: \textit{Psychological Types}, pp. 567--569. Harcourt, Brace \& Co., New York, 1946.
    \item \label{ref:ref6_poincare1948} Poincaré, H.: ``Mathematical Creation." In: J. R. Newman (Ed.), pp. 54--57. \textit{Scientific American}, CLXXIX, August 1948.
    \item \label{ref:ref7_kris1939} Kris, E.: ``On Inspiration." \textit{International Journal of Psychoanalysis}, XX:377--390, 1939.
    \item \label{ref:ref8_hutchinson1939} Hutchinson, E. D.: ``Varieties of Insight in Humans." \textit{Psychiatry}, II:323--332, 1939.
    \item \label{ref:ref9_wild1938} Wild, K. W.: \textit{Intuition}. Cambridge University Press, London, 1938.
    \item \label{ref:ref10_rhine1937} Rhine, J. B.: \textit{New Frontiers of the Mind}. Farrar \& Rinehart, New York, 1937.
    \item \label{ref:ref11_aristotle1934} Aristotle (as cited in Cohen, M. R., \& Nagel, E.): \textit{An Introduction to Logic and Scientific Method}. Harcourt, Brace \& Co., New York, 1934.
    \item \label{ref:ref12_wiener1948} Wiener, N.: \textit{Cybernetics, or Control and Communication in Animal and Machine}. John Wiley \& Sons, New York, 1948.
    \item \label{ref:ref13_hinsie1945} Hinsie, L. E., \& Shatzky, J.: See ``Perception, subconscious" in \textit{Psychiatric Dictionary}. Oxford University Press, New York, 1945.
    \item \label{ref:ref14_ronchese1945} Ronchese, F.: ``Calluses, Cicatrices and Other Stigmata as an Aid to Personal Identification." \textit{JAMA}, 128:925--931, 1945.
    \item \label{ref:ref15_deutsch1944} Deutsch, H.: \textit{Psychology of Women}, Vol. 1, p. 136. Grune \& Stratton, New York, 1944.
    \item \label{ref:ref16_pedersonkrag1947} Pederson-Krag, G.: ``Telepathy and Repression." \textit{Psychoanalytic Quarterly}, XVI:61--68, 1947.
    \item \label{ref:ref17_federn1938} Federn, P.: ``The Undirected Function in the Central Nervous System." \textit{International Journal of Psychoanalysis}, XIX:211--226, 1938.
    \item \label{ref:ref18_berne1947} Berne, E.: \textit{The Mind in Action}, pp. 279--286. Simon and Schuster, New York, 1947.
    \item \label{ref:ref19_freud1933} Freud, S.: \textit{New Introductory Lectures on Psychoanalysis}, p. 80. W. W. Norton \& Co., New York, 1933.
    \item \label{ref:ref20_wittels1945} Wittels, F.: ``Review of Stekel's \textit{Interpretation of Dreams}." \textit{Psychoanalytic Quarterly}, XIV:542, 1945.
\end{enumerate}

\end{document}